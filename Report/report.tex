% Template for ICASSP-2004 paper; to be used with:
%          spconf.sty  - ICASSP/ICIP LaTeX style file, and
%          IEEEbib.bst - IEEE bibliography style file.
% --------------------------------------------------------------------------
\documentclass{article}
\usepackage{spconf,amsmath,epsfig}

% Example definitions.
% --------------------
\def\x{{\mathbf x}}
\def\L{{\cal L}}

% Title.
% ------
\title{The application of wavelets in medical image watermarking}
%
% Single address.
% ---------------
\name{Rui Shen, Pouria Tohidi, Shuwen Wei}
\address{Johns Hopkins University}
%
% For example:
% ------------
%\address{School\\
%	Department\\
%	Address}
%
% Two addresses (uncomment and modify for two-address case).
% ----------------------------------------------------------
%\twoauthors
%  {A. Author-one, B. Author-two\sthanks{Thanks to XYZ agency for funding.}}
%	{School A-B\\
%	Department A-B\\
%	Address A-B}
%  {C. Author-three, D. Author-four\sthanks{The fourth author performed the work
%	while at ...}}
%	{School C-D\\
%	Department C-D\\
%	Address C-D}
%
\begin{document}
%\ninept
%
\maketitle
%
\begin{abstract}

\end{abstract}
%

\section{Introduction}
% Background introduction, like where the problems are
As the amount of patient data becomes increasingly larger in the hospital system, it is important to find an efficient way to store and manage the data. Traditionally, the patient data is stored separately from the corresponding medical scanned image like CT images, MRI images and so on. However, mistakes may sometimes happen when a doctor tries to extract the corresponding patient data to match a certain medical scanned image. Furthermore, the confidentiality is not always good when the hospital system is attacked by the hostile people, which may finally lead to the leakage of patient information.

% Propose a solution, review the previous work on this field
One simple solution is to embed the patient information into the corresponding scanned medical images. Watermarking, as origially used for image authentication, becomes a powerful tool in hiding the patient information into the medical scanned image. 

% Propose our method, and why it's better than the previous work or where is the innovation
In this paper, we proposed a watermarking method in wavelets domain to embed the patient data into the corresponding medical scanned image. First, we introduce the algorithm to encode the patient information into the medical image and also decode it from the image.

\section{Methods}

\section{Results}

\section{Conclusion}

\section{REFERENCES}
\label{sec:ref}

% References should be produced using the bibtex program from suitable
% BiBTeX files (here: strings, refs, manuals). The IEEEbib.bst bibliography
% style file from IEEE produces unsorted bibliography list.
% -------------------------------------------------------------------------
\bibliographystyle{IEEEbib}

\bibliography{strings,refs,manuals}


\end{document}
